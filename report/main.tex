\documentclass[10 pt, journal]{IEEEtran}

\usepackage{amsmath}
\usepackage{amssymb}
\usepackage[yyyymmdd]{datetime}
\usepackage{float}
\usepackage{graphicx}
\usepackage{listings}
\usepackage{subcaption}

\renewcommand{\dateseparator}{ - }
\renewcommand\IEEEkeywordsname{Keywords}

\title{Project in applied Mathematics \\
  \large Stitching used together with a virtual Pan, Tilt, Zoom camera}
\author{Adrian Roth and Filip Johannesson}

\markboth{\today}{}

\begin{document}

\maketitle
\begin{abstract}
	A conventional PTZ-camera may have the risk at looking in the ''wrong direction at the wrong time'', when surveying fast-paced scenes, such as an ongoing soccer game. An alternative to this is to simultaneously survey the entire scene, stitch the image together, and then creating the PTZ-motion in software.

	In this report we describe and implement such a system. Results show that for still images, the PTZ-motion and stitching works really well. 
	When applied to snapshots from real uncalibrated cameras, results were not a good, but this might have do with poor calibration rather than the stiching and PTZ-implementation.
\end{abstract}
\begin{IEEEkeywords}
	Blending, Homography, PTZ, Stitching 
\end{IEEEkeywords}
\section{Introduction}
%I love image analysis and stuff. But sometimes it just sucks. Especially these times when you're supposed to write a bloody report about stuff you probably don't understand and will just bable like you do. So enjoy:
%\\\\
%Once upon a time there was a person filming another person. The movie had to be just right for it to be rewatched year after year with many laughs and tears to follow. But there was a problem. The other person, the one being filmed, was moving very fast here and there. The person filming was having trouble getting the pan, tilt and zoom of the camera correct or else something might be missed.

%Frigöra sig från linsen
%att både ha kakan och äta upp den
%best of both worlds

\IEEEPARstart{T}{racking} a plethora of moving objects, such as an ongoing soccer game, with a mechanical Pan, Tilt, Zoom (from here on PTZ) can be a somewhat daunting task.
With things happening all over the surveyed scene and the sometimes significant response time of actuators in the PTZ-camera, there is a real possibillity that the camera is pointed in the ''wrong direction at the wrong time'', see fig. \ref{fig:problem}.

\begin{figure}[H]
	\centering
	\includegraphics[width=0.5 \columnwidth]{../results/images/PTZ_problem.jpg}
	\caption{A panable camera looking in the wrong direction, therby not noticing a moose sneaking by.}
	\label{fig:problem}
\end{figure}

In this project we investigate another method, where multiple stationary cameras are used to cover the entire scene under surveillance, and then produce the PTZ-motion in software by panoramic stitching and virtual cameras.
A conceptual depiction of the system can be seen in fig. \ref{fig:comp}.

\begin{figure}[H]
	\centering
	\includegraphics[width=0.7\columnwidth]{../results/images/PTZ_comp.png}
	\caption{A concept drawing of a virtual PTZ-camera. The input images are transformed and stitched together to form a ''curved'', stitched image of the entire scene. The virtual camera can then be seen as putting a new camera in front of this ''curved image'', where the pan and tilts are essentially rotations of the virtual camera. Zooming can be thought like magnification or reduction of the stitched image.}
	\label{fig:comp}
\end{figure}


\section{Theory}

\subsection{Speeded Up Robust Features (SURF)}
The information regarding this method is sourced from \cite{SURF}. The SURF is a method for both finding and describing interest points in an image.

\subsection{Random Sample Consensus (RANSAC)}
As described in the work of \cite{RANSAC} RANSAC is a robust method which can be applied in different applications with outliers in the data. The method uses a random component which results in non deterministic output. The output is connected to a probability and the goal is to create a high probabiliy of getting the {\it correct} output without any influence from the outliers. It is an iterative method were each iteration a the minimun number of needed data points for calculation a output, for example two points if a straight line is supposed to be calculated, is randomly choosen. Then a value of how good these points represents the real thing is used to determine if these points are the best ones to use. If the number of iterations is large the probability of getting the correct output.


\subsection{PTZ transform}
	Simply stitching together images will leave us with a large panorama image.
        However, the goal of this project is to emulate pan, tilt and zoom camera motions in post processing, so we must find a way to bring in these motions into our stitching.

	Here we introduce a virtual camera that has the same calibration parameters as the original real cameras.
	With this virtual camera we should be able to look around in the stitched image as a real camera in the real world, but how?
	The motion we seek to describe is in $\mathbb{R}^3$ and the aquired images are in $\mathbb{R}^2$, so we want to describe 3D-dimensional motion, which in this case are camera rotations.
	In order to describe perspective distortions, we introduce a third \emph{homogeneous coordinate}, thereby going from $\mathbb{R}^2$ to $\mathbb{P}^2$.
	The perspective distortion can now be described by a perspective transform matrix, carried out on each pixel in $\mathbb{P}^2$, seen in \cite{hartley2003Multiple}.
	Each pixel, with coordinates $\left( \begin{smallmatrix} x & y & w \end{smallmatrix} \right)^T$ 
	is multiplied from the left with the transformation matrix, $H_{perspective}$, and the resulting coordinates $ \left( \begin{smallmatrix} x', & y', & w' \end{smallmatrix} \right) ^T$ are the divided with homogeneous coordiante $w$. 
	Lastly the pixel coordinates are orthogonally projected down to $\left( \begin{smallmatrix} x'/w', & y'/w' \end{smallmatrix} \right)^T$ and we have returned to $\mathbb{R}^2$, see (\ref{eq:homoapp})

	\begin{multline}
		H_{perspective}\cdot
		\begin{pmatrix} 
			x \\ 
			y \\ 
			w 
		\end{pmatrix}=
		\begin{pmatrix}
			x' \\ 
			y' \\ 
			w' 
		\end{pmatrix} \rightarrow \\ 
		\rightarrow 
		\begin{pmatrix}
			x' \\ 
			y' \\ 
			w' 
		\end{pmatrix} \xrightarrow{1/w'} 
		\begin{pmatrix}
			x'/w' \\ 
			y'/w' \\ 
			w'/w' 
		\end{pmatrix} \rightarrow 
		\begin{pmatrix} 
			x'/w' \\ 
			y'/w' 
		\end{pmatrix}
		\label{eq:homoapp}
	\end{multline}

	The perspective transformation matrix, $H_{perspective}$, itself can be decomposed as in (\ref{eq:homodecomp}).
	\begin{equation}
		H_{perspective}=KR_{xyz}K^{-1}
		\label{eq:homodecomp}
	\end{equation}
	where $K$ is a calibration matrix\footnote{which transforms image points from $\mathbb{R}^3$ to $\mathbb{P}^2$} and $R_{xyz}$ is a general rotational matrix in $\mathbb{R}^3$.
$R_{xyz}$ can in turn be decomoposed into $R_x$(tilt), $R_y$(pan) and $R_z$, i.e. rotations around the x-, y-, and z-axis of the camera.
The order of these rotations are important as they determine the rotational axi of the camera. 
One could think of the rotations all occuring in their own local coordinate systems. 
If for instance the camera is rotated 10 degress along the x-axis, the next rotation will be about a new rotational axis that is tilted 10 degrees.
This means that our rotations must occur in the same order as the linkage of the PTZ we want to emulate. 
As many PTZ cameras normally employ a horizontal platform for panning on which the tilting actuators are mounted upon, we therefore first want to apply the pan rotation, (y-direction), and then applying the tilt rotation (x-direction).
However, the input images might have been captures with a tilt in relation to the pan axis.
In this case the initial tilt angle must be estimated first, and the resulting rotation applied prior to to the synthesized pan and tilt rotations.

	For the zooming we scale the images from the cameras point of view, i.e. uniform scalning in the x- and y-direction from the cameras principal point. 
	This can be implemented as a scaling matrix of the form in (\ref{eq:zoom}).
	\begin{equation}
		Z=\begin{pmatrix} 
			Zoom_x & 0 & 0 \\
			0 & Zoom_y & 0 \\
			0 & 0 & 1
		\end{pmatrix}
		\label{eq:zoom}
	\end{equation}
	where $Zoom_x$ and $Zoom_y$ are scaling factors in the x- and y- direction of the camera.
	We want our scaling to act symmetrically on the camera, i.e. want to apply the zoom when origo of our coordinate system is in the along the principal point of our camera. 
	Therefore, we apply the zoom after the rotations in (\ref{eq:homodecomp}). The resulting expression becomes (\ref{eq:finalPTZ})
	\begin{equation}
		H_{perspective}=K Z  R_{x}R_{y}  K^{-1}
		\label{eq:finalPTZ}
	\end{equation}

	We can now accurately describe 3D-rotations as transforms in $\mathbb{P}^2$.

These transforms are applied to all images after the stitching homography, as it is defined for the untransformed images.



\section{Method}

\subsection{Stitching}
This text only refers to stitching of two images, the same concept can be expanded for multiple images. The stitching can be divided into two parts. The first one is to find a relation, homography, between two images. In other words where an image {\it fits} in the other image. The other part is to combine or blend the images to make the transition between them as smooth as possible.
\\\\
To find a relation between two images, mutual points have to be found. This is done using a point detector. The detector should find points of interest. Then the points needs a value for the possibility of matching points that are similar. Here a feature descriptor is needed. In this project a method called SURF (Speeded Up Robust Features) has been used. It includes both a detector and descriptor. The points and descriptors in each image are matched and it is very likely that they contain outliers. However, the RANSAC method can be used to avoid influence of the outliers when calculating the homography between the images.
\\\\
The blending of the two images consists of three steps. First the previously calculated homogaphy is used to get the images in the same perspective. Here there is an overlap between the images. This overlap is weighted with a value ($w_1 \in [0, 1]$) for the first image and ($w_2 = 1 - w_1$) for the other. The weighting function is choosen to get the smoothest transition.


\subsection{Virtual PTZ camera}
Using the decomposition in (\ref{eq:homodecomp}) we can transform the problem into three rotations, where we only really need to use two of them, the panning rotation around the camera Y-axis and rotation around the camera X-axis after the panning has been applied.
The rotational matrices are implemented as follows in (\ref{eq:pan}) and (\ref{eq:tilt})
	\begin{align}		
		R_{pan}&=\begin{pmatrix} 
			\cos(\theta_y) & 0 & -\sin(\theta_y) \\
			0 & 1 & 0 \\
			\sin(\theta_y) & 0 & \cos(\theta_y)
		\end{pmatrix} \label{eq:pan} \\
		R_{tilt} &=\begin{pmatrix}
			1 & 0 & 0 \\
			0 & \cos(\theta_x) & -\sin(\theta_x) \\
			0 & \sin(\theta_x) & \cos(\theta_x)
		\end{pmatrix} \label{eq:tilt}
	\end{align}
	where $\theta_x$ is the tilt angle and $\theta_y$ is the pan angle, in radians.
	For the particular input images in fig. \ref{fig:input}, the cameras were slighty tilted , with an estimated initial tilt angle of $\approx 0.47$ radians.\footnote{The exact estimated angle was 0.47179832679 radians.}
	This means that if we want our panning axis to coincide with the panning axis of the input images, we first need to apply a tilting of $-0.47$ radians, and then apply our panning matrix, (\ref{eq:pan}) and then our tilting matrix, \ref{eq:tilt}. 
	The ''initial tilt correction'' can be written in matrix form as (\ref{eq:tiltinit}).
	\begin{multline}
		R_{tiltCorr}=\begin{pmatrix}
			1 & 0 & 0 \\
			0 & \cos(-\theta_{tiltinit}) & -\sin(-\theta_{tiltinit}) \\
			0 & \sin(-\theta_{tiltinit}) & \cos(-\theta_{tiltinit})
		\end{pmatrix} = \\
		=\begin{pmatrix}
			1 & 0 & 0 \\
			0 & \cos(\theta_{tiltinit}) & \sin(\theta_{tiltinit}) \\
			0 & -\sin(\theta_{tiltinit}) & \cos(\theta_{tiltinit})
		\end{pmatrix}
		\label{eq:tiltinit}
	\end{multline}
	where $\theta_{tiltinit} \approx 0.47$ radians.

	The zoom is implemented as stated in (\ref{eq:zoom}).
	The matrices are finally multiplied together to a single composite matrix, $H_{perspective}$, see (\ref{eq:PTZcomp})

	\begin{equation}
		H_{perspective}=KZR_{tilt}R_{pan}R_{tiltCorr}K^{-1}
		\label{eq:PTZcomp}
	\end{equation}
	where K are the camera calibration matrices.
	As stated in the theory section, the homographies applied on the images are variants of a composite matrix defined as in (\ref{eq:comp}).

	\begin{equation}
		H_{composite}=H_{perspective}H_{stitching}
		\label{eq:comp}
	\end{equation}
	Where $H_{stitching}$ is the homography produced by the stitching algorithm, invidiual for each image. Note that the stitching transform is applied prior to the perspective transforms. % is it not the other way around



\section{Results}

\subsection{Stitching}
The stitching was performed with two different blending functions. The result of stitching image one and two with a linear transision is shown in figure \ref{fig:results:stitching:linear} and the corresponding one using a sigmoid function is shown in figure \ref{fig:results:stitching:sigmoid}. A stitching performed when using two cameras in a home made set-up is found in figure \ref{fig:results:stitching:homemade}.

\begin{figure}[H]
  \centering
  \includegraphics[width = 0.9\columnwidth]{../results/stitch_linear.jpg}
  \caption{Stitching done using a linear transition.}
  \label{fig:results:stitching:linear}
\end{figure}

\begin{figure}[H]
  \centering
  \includegraphics[width = 0.9\columnwidth]{../results/stitch_sigmoid.jpg}
  \caption{Stitching done using a sigmoid transition.}
  \label{fig:results:stitching:sigmoid}
\end{figure}

\begin{figure*}
	\centering
	\begin{subfigure}[t]{0.3\textwidth}
		\centering
		\includegraphics[width=\textwidth]{../data/homography/camera_1.jpg}
		\caption{input 1}
	\end{subfigure}
	\begin{subfigure}[t]{0.3\textwidth}
		\centering
		\includegraphics[width=\textwidth]{../data/homography/camera_2.jpg}
		\caption{input 2}
	\end{subfigure}
		\begin{subfigure}[t]{0.3\textwidth}
		\centering
                \includegraphics[width = \textwidth]{../results/home_made_stitch.jpg}
		\caption{output}
	\end{subfigure}
        \caption{Stitching done with home made set-up.}
	\label{fig:results:stitching:homemade}
\end{figure*}


\subsection{Virtual PTZ camera}
Initially, some simple transformations were carried out in MatLab using a symmetrical set of normalized homogeneous points, i.e. four points with coordinates (-1,-1,1), (1,-1,1), (1,1,1) and (-1,1,1) in $\mathbb{P}^2$. 

Secondary trials were made with a point set with origo in a corner point, similar to the coordinate system in an image. Here ``calibration'' matrices were needed in order to move the image origo to the center of the image and scale it. 

Advancments into real-images were then straight forward as the synthezised calibration matrices were replaced with the real calibration matrices. 

Actual implementation in c++ in combination with the stiching homographies were initially troublesome, but as it was soon realized that first applying stiching homographies and then applying the PTZ-transform made the images keep the correct relations to each other. The were also some problems with the PTZ-transform being applied twice on some of the stiched images, giving the impression that parts of the scene were ``drifting'' away, as well as the image masks not being transformed the same way as the images themselves.

These issues were however solved and an example of the resulting ptz movement can be seen in fig. \ref{fig:ptz_res}.

\begin{figure}[H]
	\centering
	\includegraphics[width=0.8\columnwidth]{../results/images/PTZ_res.PNG}
	\caption{An example of the resulting image where the virtual camera has been panned, tilted and zoomed.}
	\label{fig:ptz_res}
\end{figure}



\section{Discussion}
As seen in figures \ref{fig:results:stitching:linear} and \ref{fig:results:stitching:sigmoid} the stitching works pretty good for both blending functions. The most important here is to calculate a good homography. However, as seen the images were not taken simultaneously which is clear in the linear blending. Therefore the sigmoid blending is probably preferred for this application. Calculation of the homogaphy is not as robust as one would want. Especially seen in figure \ref{fig:results:stitching:sigmoid}. Here the images have been stitched using two {\it real} cameras with homemade calibration. From this result possible improvements in either the stitching or calibration is needed.


\bibliography{citations.bib}
\bibliographystyle{apalike}

\end{document}
