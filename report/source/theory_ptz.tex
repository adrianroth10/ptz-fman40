\subsection{PTZ transform}
	Simply stitching together images will leave us with a large panorama image. However, the goal of this project is to emulate pan, tilt and zoom camera motions in post processing, so we must find a way to bring in these motions into our stitching.

	Here we introduce a virtual camera that has the same calibration parameters as the original real cameras. 
	With this virtual camera we should be able to look around in the stitched image as a real camera in the real world, but how? 
	The motion we seek to describe is in $\mathbb{R}^3$ and the aquired images are in $\mathbb{R}^2$, so we want to describe 3D-dimensional motion\footnote{which in this case is mainly rotations} from the reference point of the projected images. 
	In order to describe perspective distortions, we introduce a third \emph{homogeneous coordinate}, thereby going from $\mathbb{R}^2$ to $\mathbb{P}^2$. 
	The perspective distortion can now be described by a perspective transform matrix, carried out on each pixel in $\mathbb{P}^2$, seen in \cite{hartley2003Multiple}.

	The perspective transformation matrix, $H_{perspective}$, itself can be decomposed as in (\ref{eq:homodecomp}).
	\begin{equation}
		H_{perspective}=KR_{xyz}K^{-1}
		\label{eq:homodecomp}
	\end{equation}
	where $K$ is a calibration matrix\footnote{which transforms image points from $\mathbb{R}^3$ to $\mathbb{P}^2$} and $R_{xyz}$ is a general rotational matrix in $\mathbb{R}^3$.
$R_{xyz}$ can in turn be decomoposed into $R_x$, $R_y$ and $R_z$, i.e. rotations around the x-, y-, and z-axis from the initial pose of the image.

	The zooming part of the camera motion can be described by a simple scaling matrix in $\mathbb{R}^3$ as in, $a_{zoom}I$, where $a_{zoom}$ is a scaling factor and $I$ is an $3 \times 3$ identity matrix.\footnote{As the images are flat planes in $\mathbb{R}^3$ the last entry of the matrix has little effect.}

	We can now accurately describe 3D-rotations as transforms in $\mathbb{P}^2$. For the zoom we simply scale the images.
	We also introduce a small translation of the images in $P^2$ on the form seen in (\ref{eq:trans}). 
	This is applied last mainly to slightly improve the panning motion of the virtual camera. 
	$t_x$ and $t_y$ are translations in x- and y-directions respectively.
\begin{equation}
	T=\begin{pmatrix}
		1 & 0 & t_x \\
		0 & 1 & t_y \\
		0 & 0 & 1
	\end{pmatrix}
	\label{eq:trans}
\end{equation}

These transforms are applied to all images after the stitching homography, as it is defined for the untransformed images.
