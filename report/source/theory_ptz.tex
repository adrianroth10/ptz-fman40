\subsection{PTZ transform}
	Simply sitiching together images will leave us with a large panorama image. However, the goal of this project is to emulate pan, tilt and zoom camera motions in post processing, so we must find a way to bring in these motions into our stitching. 

	Here we introduce a virtual camera that has the same calibration parameters as the original real cameras. With this virtual camera we should be able to look around in the stiched image as a real camera in the real world, but how? The motion we seek do describe is in $\mathbb{R}^3$ and the aquired images are in $\mathbb{R}^2$, so we want to describe 3D-dimensional motion\footnote{which in this case is mainly rotations} from the reference point of the projected images. In order to describe perspective distortions, we introduce a third \emph{homogeneous coordinate}, thereby going from $\mathbb{R}^2$ to $\mathbb{P}^2$. The perspective distortion can now be desribed by a perspective transform matrix, carried out on each pixel in $\mathbb{P}^2$.

	The perspective tranformation matrix, $H_{perspective}$, itself can be decomposed as in eq. (\ref{eq:homodecomp}) 
	\begin{equation}
		H_{perspective}=KR_{xyz}K^{-1}
		\label{eq:homodecomp}
	\end{equation}
	where $K$ is the camera matrix\footnote{which transforms imagepoints from $\mathbb{R}^3$ to $\mathbb{P}^2$} and $R_{xyz}$ is a general rotational matrix in $\mathbb{R}^3$. 
$R_{xyz}$ can in turn be decomoposed into $R_x$, $R_y$ and $R_z$, i.e. rotations around the x-, y-, and z-axis from the initial pose of the image.

	The zooming part of the camera motion can be described by a simple scaling matrix in $\mathbb{R}^3$ as in, $a_{zoom}*I$, where $a_{zoom}$ is a scaling factor and $I$ is an $3 \times 3$ identity matrix. \footnote{As the images can be considered planes in the $\mathbb{R}^3$ the last entry of the matrix has little effect.}

	We can now accurately describe 3D-rotations as transforms in $\mathbb{P}^2$. For the zoom we simply scale the images. This tranform is applied to all images before the Stitching homography. So one could see it as a two stage process, where we first adjust the images according the to the virtual camera and then adjust some of the images in order to perform the stiching. 
