\subsection{Virtual PTZ camera}
Using the decomposition in (\ref{eq:problem}) we can transform the problem into three rotations, where we only really need to use two of them, rotation around the image X-axis and rotation around the image Y-axis. The rotational matrices are implemented as follows in (\ref{eq:pan}) and (\ref{eq:tilt})
	\begin{align}
		R_x &=\begin{pmatrix}1 & 0 & 0 \\
			0 & \cos(\theta_x) & -\sin(\theta_x) \\
			0 & \sin(\theta_x) & \sin(\theta_x)
		\end{pmatrix} \label{eq:pan}\\ 
		R_y&=\begin{pmatrix} \cos(\theta_y) & 0 & \sin(\theta_y) \\
			0 & 1 & 0 \\
			\cos(\theta_y) & 0 & \sin(\theta_y)
		\end{pmatrix} \label{eq:tilt}
	\end{align}

	Note that these rotations are around the image axi and not around the camera center. In order to emulate rotation around the camera center, a translation matrix is also applied, defined in $\mathbb{P}^2$ as (\ref{eq:zmtrns}). Zooming is also applied in this matrix.

	\begin{equation}
		T=\begin{pmatrix} 
			Zoom_x & 0 & T_x \\ 
			0 & Zoom_y & T_y \\ 
			0 & 0 & 1 
		\end{pmatrix} \label{eq:zmtrns}
	\end{equation}
Where $Zoom_x$ and $Zoom_y$ are scale factors in the x- and y-direction, but since we scale the image uniformly, these are the same. $T_x$ and $T_y$ are translations in the x- and y-direction.
