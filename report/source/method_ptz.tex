\subsection{Virtual PTZ camera}
Using the decomposition in (\ref{eq:homodecomp}) we can transform the problem into three rotations, where we only really need to use two of them, rotation around the image X-axis and rotation around the image Y-axis.
The rotational matrices are implemented as follows in (\ref{eq:pan}) and (\ref{eq:tilt})
	\begin{align}
		R_x &=\begin{pmatrix}1 & 0 & 0 \\
			0 & \cos(\theta_x) & -\sin(\theta_x) \\
			0 & \sin(\theta_x) & \cos(\theta_x)
		\end{pmatrix} \label{eq:pan}\\
		R_y&=\begin{pmatrix} \cos(\theta_y) & 0 & -\sin(\theta_y) \\
			0 & 1 & 0 \\
			\sin(\theta_y) & 0 & \cos(\theta_y)
		\end{pmatrix} \label{eq:tilt}
	\end{align}

	Note that these rotations are around the image axis and not around the camera center.
        In order to emulate rotation around the camera center, a translation matrix is also applied, defined in $\mathbb{P}^2$ as (\ref{eq:zmtrns}).

	\begin{equation}
		T=\begin{pmatrix}
			1 & 0 & T_x \\
			0 & 1 & T_y \\
			0 & 0 & 1
		\end{pmatrix} \label{eq:zmtrns}
	\end{equation}
	$T_x$ and $T_y$ are translations in the x- and y-direction.
        $T_x$ is partly regulated by the tilt angle and a tilt angle offest from a completly horizontal camera pose.
	The zoom is implemented as (\ref{eq:zoom}).
	\begin{equation}
		Z=\begin{pmatrix}
			Zoom_x & 0 & 0\\
			0 & Zoom_y & 0\\
			0 & 0 & 1
		\end{pmatrix}
		\label{eq:zoom}
	\end{equation}

	Where $Zoom_x$ and $Zoom_y$ are scale factors in the x- and y-direction, but since we scale the image uniformly, these are the same.

	The matrices are finally multiplied together to a single composite matrix, $H_{perspective}$, see (\ref{eq:PTZcomp})

	\begin{equation}
		H_{perspective}=TKZR_yR_xK^{-1}
		\label{eq:PTZcomp}
	\end{equation}
	where K are the camera calibration matrices.
	As stated in the theory section, the homographies applied on the images are variants of a composite matrix defined as in (\ref{eq:comp}).

	\begin{equation}
		H_{composite}=H_{perspective}H_{stitching}
		\label{eq:comp}
	\end{equation}
	Where $H_{stitching}$ is the homography produced by the stitching algorithm, invidiual for each image. Note that the stitching transform is applied prior to the perspective transforms. % is it not the other way around
