\subsection{Virtual PTZ camera}
Using the decomposition in (\ref{eq:homodecomp}) we can transform the problem into three rotations, where we only really need to use two of them, the panning rotation around the camera Y-axis and rotation around the camera X-axis after the panning has been applied.
The rotational matrices are implemented as follows in (\ref{eq:pan}) and (\ref{eq:tilt})
	\begin{align}		
		R_{pan}&=\begin{pmatrix} 
			\cos(\theta_y) & 0 & -\sin(\theta_y) \\
			0 & 1 & 0 \\
			\sin(\theta_y) & 0 & \cos(\theta_y)
		\end{pmatrix} \label{eq:pan} \\
		R_{tilt} &=\begin{pmatrix}
			1 & 0 & 0 \\
			0 & \cos(\theta_x) & -\sin(\theta_x) \\
			0 & \sin(\theta_x) & \cos(\theta_x)
		\end{pmatrix} \label{eq:tilt}
	\end{align}
	where $\theta_x$ is the tilt angle and $\theta_y$ is the pan angle, in radians.
	For the particular input images in fig. \ref{fig:input}, the cameras were slighty tilted , with an estimated initial tilt angle of $\approx 0.47$ radians.\footnote{The exact estimated angle was 0.47179832679 radians.}
	This means that if we want our panning axis to coincide with the panning axis of the input images, we first need to apply a tilting of $-0.47$ radians, and then apply our panning matrix, (\ref{eq:pan}) and then our tilting matrix, \ref{eq:tilt}. 
	The ''initial tilt correction'' can be written in matrix form as (\ref{eq:tiltinit}).
	\begin{multline}
		R_{tiltCorr}=\begin{pmatrix}
			1 & 0 & 0 \\
			0 & \cos(-\theta_{tiltinit}) & -\sin(-\theta_{tiltinit}) \\
			0 & \sin(-\theta_{tiltinit}) & \cos(-\theta_{tiltinit})
		\end{pmatrix} = \\
		=\begin{pmatrix}
			1 & 0 & 0 \\
			0 & \cos(\theta_{tiltinit}) & \sin(\theta_{tiltinit}) \\
			0 & -\sin(\theta_{tiltinit}) & \cos(\theta_{tiltinit})
		\end{pmatrix}
		\label{eq:tiltinit}
	\end{multline}
	where $\theta_{tiltinit} \approx 0.47$ radians.

	The zoom is implemented as stated in (\ref{eq:zoom}).
	The matrices are finally multiplied together to a single composite matrix, $H_{perspective}$, see (\ref{eq:PTZcomp})

	\begin{equation}
		H_{perspective}=KZR_{tilt}R_{pan}R_{tiltCorr}K^{-1}
		\label{eq:PTZcomp}
	\end{equation}
	where K are the camera calibration matrices.
	As stated in the theory section, the homographies applied on the images are variants of a composite matrix defined as in (\ref{eq:comp}).

	\begin{equation}
		H_{composite}=H_{perspective}H_{stitching}
		\label{eq:comp}
	\end{equation}
	Where $H_{stitching}$ is the homography produced by the stitching algorithm, invidiual for each image. Note that the stitching transform is applied prior to the perspective transforms. % is it not the other way around
