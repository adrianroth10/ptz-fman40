\section{Virtual PTZ camera}
Initially, some simple transformations were carried out in MatLab using a symmetrical set of normalized homogeneous points, i.e. four points with coordinates (-1,-1,1), (1,-1,1), (1,1,1) and (-1,1,1) in $\mathbb{P}^2$. 

Secondary trials were made with a point set with origo in a corner point, similar to the coordinate system in an image. Here ``camera'' matrices were needed in order to move the image origo to the center of the image. 

Advancments into real-images were then straight forward as the sythesied cmaera matrices were replaced with the real matrices. 

Actual implementation in c++ in combination with the stiching homographies, were initially troublesome, but as it was soon realized that first applying stiching homographies and then applying the PTZ-transform made the images keep the correct relations to each other. The were also some problems with the PTZ-transform being applied twice on some of the stiched images, giving the impression that parts of the scene were ``drifting'' away, as well as the image masks not being tranfromed the same way as the images themselves.

These issues were however solved and the results can be seen in figures to come!
