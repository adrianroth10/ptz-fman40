\section{Discussion}
For the stitching there two different blending functions were used as seen in figures \ref{fig:results:stitching:linear} and \ref{fig:results:stitching:sigmoid}. The blending first of all reveals that the images are not taken at the same time. This means that the scene has changed betwen the images. With this in mind the sigmoid function gives a better result, a smoother transition in the stitching. Otherwise the stitching has prooved to be not so robust as one would want. During some part of the project the homography calculation between the images were wrong. But twitching some parameter in opencv's function findHomography fixed this. Then for our home made camera set-up the stitching performed quite poorly as seen in figure \ref{fig:results:stitching:homemade}. Except for the possibility of bad algorithms the result might also be improved by doing a better calibration of the cameras.
